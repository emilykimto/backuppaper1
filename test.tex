% Options for packages loaded elsewhere
\PassOptionsToPackage{unicode}{hyperref}
\PassOptionsToPackage{hyphens}{url}
\PassOptionsToPackage{dvipsnames,svgnames,x11names}{xcolor}
%
\documentclass[
  letterpaper,
  DIV=11,
  numbers=noendperiod]{scrartcl}

\usepackage{amsmath,amssymb}
\usepackage{lmodern}
\usepackage{iftex}
\ifPDFTeX
  \usepackage[T1]{fontenc}
  \usepackage[utf8]{inputenc}
  \usepackage{textcomp} % provide euro and other symbols
\else % if luatex or xetex
  \usepackage{unicode-math}
  \defaultfontfeatures{Scale=MatchLowercase}
  \defaultfontfeatures[\rmfamily]{Ligatures=TeX,Scale=1}
\fi
% Use upquote if available, for straight quotes in verbatim environments
\IfFileExists{upquote.sty}{\usepackage{upquote}}{}
\IfFileExists{microtype.sty}{% use microtype if available
  \usepackage[]{microtype}
  \UseMicrotypeSet[protrusion]{basicmath} % disable protrusion for tt fonts
}{}
\makeatletter
\@ifundefined{KOMAClassName}{% if non-KOMA class
  \IfFileExists{parskip.sty}{%
    \usepackage{parskip}
  }{% else
    \setlength{\parindent}{0pt}
    \setlength{\parskip}{6pt plus 2pt minus 1pt}}
}{% if KOMA class
  \KOMAoptions{parskip=half}}
\makeatother
\usepackage{xcolor}
\setlength{\emergencystretch}{3em} % prevent overfull lines
\setcounter{secnumdepth}{-\maxdimen} % remove section numbering
% Make \paragraph and \subparagraph free-standing
\ifx\paragraph\undefined\else
  \let\oldparagraph\paragraph
  \renewcommand{\paragraph}[1]{\oldparagraph{#1}\mbox{}}
\fi
\ifx\subparagraph\undefined\else
  \let\oldsubparagraph\subparagraph
  \renewcommand{\subparagraph}[1]{\oldsubparagraph{#1}\mbox{}}
\fi


\providecommand{\tightlist}{%
  \setlength{\itemsep}{0pt}\setlength{\parskip}{0pt}}\usepackage{longtable,booktabs,array}
\usepackage{calc} % for calculating minipage widths
% Correct order of tables after \paragraph or \subparagraph
\usepackage{etoolbox}
\makeatletter
\patchcmd\longtable{\par}{\if@noskipsec\mbox{}\fi\par}{}{}
\makeatother
% Allow footnotes in longtable head/foot
\IfFileExists{footnotehyper.sty}{\usepackage{footnotehyper}}{\usepackage{footnote}}
\makesavenoteenv{longtable}
\usepackage{graphicx}
\makeatletter
\def\maxwidth{\ifdim\Gin@nat@width>\linewidth\linewidth\else\Gin@nat@width\fi}
\def\maxheight{\ifdim\Gin@nat@height>\textheight\textheight\else\Gin@nat@height\fi}
\makeatother
% Scale images if necessary, so that they will not overflow the page
% margins by default, and it is still possible to overwrite the defaults
% using explicit options in \includegraphics[width, height, ...]{}
\setkeys{Gin}{width=\maxwidth,height=\maxheight,keepaspectratio}
% Set default figure placement to htbp
\makeatletter
\def\fps@figure{htbp}
\makeatother
\newlength{\cslhangindent}
\setlength{\cslhangindent}{1.5em}
\newlength{\csllabelwidth}
\setlength{\csllabelwidth}{3em}
\newlength{\cslentryspacingunit} % times entry-spacing
\setlength{\cslentryspacingunit}{\parskip}
\newenvironment{CSLReferences}[2] % #1 hanging-ident, #2 entry spacing
 {% don't indent paragraphs
  \setlength{\parindent}{0pt}
  % turn on hanging indent if param 1 is 1
  \ifodd #1
  \let\oldpar\par
  \def\par{\hangindent=\cslhangindent\oldpar}
  \fi
  % set entry spacing
  \setlength{\parskip}{#2\cslentryspacingunit}
 }%
 {}
\usepackage{calc}
\newcommand{\CSLBlock}[1]{#1\hfill\break}
\newcommand{\CSLLeftMargin}[1]{\parbox[t]{\csllabelwidth}{#1}}
\newcommand{\CSLRightInline}[1]{\parbox[t]{\linewidth - \csllabelwidth}{#1}\break}
\newcommand{\CSLIndent}[1]{\hspace{\cslhangindent}#1}

\usepackage{booktabs}
\usepackage{longtable}
\usepackage{array}
\usepackage{multirow}
\usepackage{wrapfig}
\usepackage{float}
\usepackage{colortbl}
\usepackage{pdflscape}
\usepackage{tabu}
\usepackage{threeparttable}
\usepackage{threeparttablex}
\usepackage[normalem]{ulem}
\usepackage{makecell}
\usepackage{xcolor}
\KOMAoption{captions}{tableheading}
\makeatletter
\makeatother
\makeatletter
\makeatother
\makeatletter
\@ifpackageloaded{caption}{}{\usepackage{caption}}
\AtBeginDocument{%
\ifdefined\contentsname
  \renewcommand*\contentsname{Table of contents}
\else
  \newcommand\contentsname{Table of contents}
\fi
\ifdefined\listfigurename
  \renewcommand*\listfigurename{List of Figures}
\else
  \newcommand\listfigurename{List of Figures}
\fi
\ifdefined\listtablename
  \renewcommand*\listtablename{List of Tables}
\else
  \newcommand\listtablename{List of Tables}
\fi
\ifdefined\figurename
  \renewcommand*\figurename{Figure}
\else
  \newcommand\figurename{Figure}
\fi
\ifdefined\tablename
  \renewcommand*\tablename{Table}
\else
  \newcommand\tablename{Table}
\fi
}
\@ifpackageloaded{float}{}{\usepackage{float}}
\floatstyle{ruled}
\@ifundefined{c@chapter}{\newfloat{codelisting}{h}{lop}}{\newfloat{codelisting}{h}{lop}[chapter]}
\floatname{codelisting}{Listing}
\newcommand*\listoflistings{\listof{codelisting}{List of Listings}}
\makeatother
\makeatletter
\@ifpackageloaded{caption}{}{\usepackage{caption}}
\@ifpackageloaded{subcaption}{}{\usepackage{subcaption}}
\makeatother
\makeatletter
\@ifpackageloaded{tcolorbox}{}{\usepackage[many]{tcolorbox}}
\makeatother
\makeatletter
\@ifundefined{shadecolor}{\definecolor{shadecolor}{rgb}{.97, .97, .97}}
\makeatother
\makeatletter
\makeatother
\ifLuaTeX
  \usepackage{selnolig}  % disable illegal ligatures
\fi
\IfFileExists{bookmark.sty}{\usepackage{bookmark}}{\usepackage{hyperref}}
\IfFileExists{xurl.sty}{\usepackage{xurl}}{} % add URL line breaks if available
\urlstyle{same} % disable monospaced font for URLs
\hypersetup{
  pdftitle={Evaluating the Necessity of Language Translation Services for Emergency Dispatch Calls in Toronto},
  pdfauthor={Emily Kim},
  colorlinks=true,
  linkcolor={blue},
  filecolor={Maroon},
  citecolor={Blue},
  urlcolor={Blue},
  pdfcreator={LaTeX via pandoc}}

\title{Evaluating the Necessity of Language Translation Services for
Emergency Dispatch Calls in Toronto\thanks{Code and data are available
at https://github.com/emilykimto/Language-Services.git}}
\usepackage{etoolbox}
\makeatletter
\providecommand{\subtitle}[1]{% add subtitle to \maketitle
  \apptocmd{\@title}{\par {\large #1 \par}}{}{}
}
\makeatother
\subtitle{How We Might Make Language Services More Accessible to
Toronto's Diverse Communities}
\author{Emily Kim}
\date{1 February 2023}

\begin{document}
\maketitle
\begin{abstract}
Toronto is a city known for its diverse immigrant communities, resulting
in a variety of non-English first languages being spoken. This diversity
can create challenges for emergency services to provide efficient
support in time-sensitive situations. This report aims to examine the
usage of designated language translation services by Toronto Paramedic
Services' Emergency Medical Dispatchers during 911 calls from 2014 to
2021. By analyzing this data, we have discovered that Cantonese,
Mandarin, and Russian are the top three languages in need of improved
accessibility to specific language services during emergencies, with
Spanish on an upward trend.
\end{abstract}
\ifdefined\Shaded\renewenvironment{Shaded}{\begin{tcolorbox}[boxrule=0pt, sharp corners, interior hidden, borderline west={3pt}{0pt}{shadecolor}, frame hidden, breakable, enhanced]}{\end{tcolorbox}}\fi

\hypertarget{introduction}{%
\section{Introduction}\label{introduction}}

Toronto is Canada's most diverse city, with nearly half of the city's
population being immigrants from all around the world. Within this urban
diversity are people from countries where the dominant language is not
English. One important consequence of this phenomenon is the inability
for these individuals to clearly express their crisis in an emergency
situation, such as during a 911 call. Communication difficulties can
lead to misunderstandings, and the caller may not be able to provide all
the necessary information to the operator which can result in confusion
and delay. Particularly in the event that they require health services,
linguistic barriers can result in a life-threatening situation. Recent
research studies by linguists show that during the COVID-19 global
pandemic, the integration of emergency language services in the COVID-19
response helped greatly reduce the spread of the virus (Dreisbach and
Mendoza-Dreisbach 2020). Similarly, for an ethnically diverse city like
Toronto, being able to accommodate those with limited English
proficiency can help minimize the potential harm in emergency
situations. Language interpreting services such as those offered by
Toronto's Paramedic Services 911 are crucial to ensuring accurate
communication, timely and effective delivery of emergency services, and
improving outcomes.

As Toronto experiences a growing immigrant population, there is an
increasing demand for emergency interpretation services that can
accommodate the diverse linguistic backgrounds of the city's residents.
This report aims to explore this need by analyzing the City of Toronto's
paramedic language services data to uncover which communities have the
most demand for emergency interpretation services. I will determine the
top languages requiring the most cumulative sum of interpretation
services duration (in minutes) between 2014-2021. Then, I will compare
the top languages with highest call duration results with fluctuations
over the eight year period to uncover a more recent trend in language
usage.

By examining the trend of language-specific calls between 2014-2021, we
can gain a deeper understanding of the city's evolving linguistic needs.
This information can then inform the allocation of resources, such as
staff and time, to improve the quality and accessibility of emergency
services for Toronto's many ethnic communities.

\hypertarget{data}{%
\section{Data}\label{data}}

The data used in this report was collected by the Toronto Paramedic
Services (Data 2022) from the Toronto Open Data portal (Gelfand 2020).
It was last refreshed on February 2, 2022, and updates yearly. The data
set contains information on the date, time, language, and duration in
minutes of instances when a caller required language interpretation.
Using \texttt{R} (R Core Team 2021), \texttt{tidyverse} (Wickham et al.
2019), and \texttt{dplyr} (Wickham et al. 2021), I began my analysis.

First, I grouped the data by language using \texttt{group\_by} before
using the function \texttt{summarize} from the \texttt{dplyr} package to
calculate the sum of duration for each language presented in the data.
This resulted in a new data frame with one row for each language paired
with the sum of all durations for that language. With this new data
frame, I renamed the ``total\_duration'' column to ``Total Duration''
using the \texttt{rename()} function for further clarity. I wanted to
find the top ten languages in the data set that had the highest total
call duration, so I used the \texttt{top\_n} function after which I
arranged the data by ``Total Duration'' in descending order using
\texttt{arrange}. This created a final table consisting of the top ten
languages requiring the most language interpretation assistance during
emergency calls using \texttt{knitr} (Xie 2021) and \texttt{kableExtra}
(Zhu 2020), please refer to Table~\ref{tbl-toplanguages}.

\newpage

\hypertarget{tbl-toplanguages}{}
\begin{table}

\caption{\label{tbl-toplanguages}Top 10 languages based on the total duration of calls requiring language
services }Top 10 languages based on the total duration of calls requiring language services}
\centering
\begin{tabu} to \linewidth {>{\raggedright\arraybackslash}p{12cm}>{\raggedright}X}
\toprule
Language & Duration\\
\midrule
ARABIC & 8647\\
CANTONESE & 24102\\
FARSI & 9181\\
HUNGARIAN & 9292\\
ITALIAN & 10238\\
\addlinespace
MANDARIN & 22884\\
PORTUGUESE & 8994\\
RUSSIAN & 17384\\
SPANISH & 14638\\
TAMIL & 11016\\
\bottomrule
\end{tabu}
\end{table}

\hypertarget{refs}{}
\begin{CSLReferences}{1}{0}
\leavevmode\vadjust pre{\hypertarget{ref-languageservices}{}}%
Data, Toronto Open. 2022. {``Paramedic Services 911 Language
Interpretation.''}
\url{https://open.toronto.ca/dataset/paramedic-services-911-language-interpretation/}.

\leavevmode\vadjust pre{\hypertarget{ref-Dreisbach2020}{}}%
Dreisbach, Jeconiah Louis, and Sharon Mendoza-Dreisbach. 2020. {``The
Integration of Emergency Language Services in {COVID}-19 Response: A
Call for the Linguistic Turn in Public Health.''} \emph{Journal of
Public Health} 43 (2): e248--49.
\url{https://doi.org/10.1093/pubmed/fdaa178}.

\leavevmode\vadjust pre{\hypertarget{ref-opendatatoronto}{}}%
Gelfand, Sharla. 2020. \emph{Opendatatoronto: Access the City of Toronto
Open Data Portal}.
\url{https://CRAN.R-project.org/package=opendatatoronto}.

\leavevmode\vadjust pre{\hypertarget{ref-citeR}{}}%
R Core Team. 2021. \emph{R: A Language and Environment for Statistical
Computing}. Vienna, Austria: R Foundation for Statistical Computing.
\url{https://www.R-project.org/}.

\leavevmode\vadjust pre{\hypertarget{ref-citetidyverse}{}}%
Wickham, Hadley, Mara Averick, Jennifer Bryan, Winston Chang, Lucy
D'Agostino McGowan, Romain François, Garrett Grolemund, et al. 2019.
{``Welcome to the {tidyverse}.''} \emph{Journal of Open Source Software}
4 (43): 1686. \url{https://doi.org/10.21105/joss.01686}.

\leavevmode\vadjust pre{\hypertarget{ref-dplyr}{}}%
Wickham, Hadley, Romain François, Lionel Henry, and Kirill Müller. 2021.
\emph{Dplyr: A Grammar of Data Manipulation}.
\url{https://CRAN.R-project.org/package=dplyr}.

\leavevmode\vadjust pre{\hypertarget{ref-knitr}{}}%
Xie, Yihui. 2021. \emph{Knitr: A General-Purpose Package for Dynamic
Report Generation in r}. \url{https://yihui.org/knitr/}.

\leavevmode\vadjust pre{\hypertarget{ref-kableExtra}{}}%
Zhu, Hao. 2020. \emph{kableExtra: Construct Complex Table with 'Kable'
and Pipe Syntax}. \url{https://CRAN.R-project.org/package=kableExtra}.

\end{CSLReferences}



\end{document}
